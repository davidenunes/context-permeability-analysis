\documentclass[11pt]{letter} 

\begin{document}

\textbf{Journal: Artificial Intelligence\\
Ref. No. ARTINT-D-13-00131R1\\
Title: Modelling Structured Societies: a Multi-relational Approach to Context Permeability\\
}

Answer to the reviewer's comments:

\begin{itemize}
\item Reviewer 1 is satisfied with the improvements we made over the first submitted version.

\item Reviewer 2 raises two very valid points:
\begin{itemize}
\item[--2a] s/he argues that our scenario is too specific and of little use for a general AI practitioner, the motivation needs to be stronger. 
\item[--2b] s/he misses a summary of results and explanation about their relevance.
\end{itemize}

\item Reviewer 3:
\begin{itemize}
\item[--3a] has doubts about applicability, and sustains that the models we use are too detached from reality, especially when concerning the understanding of human society.
\item [--3b] and mentions some typos and grammar errors.
\end{itemize}

\end{itemize}

In the new version of our paper we have included the following changes.

\begin{itemize}
\item[A:] In order to better explain the motivation of the paper and applicability of the work (that Reviewer 1 finds good, but Reviewers 2 and 3 still considered that required further clarification), and in this way to address comments 2a and 3a, we have made substantial changes to the introduction, namely in pointing that the emphasis is not really on the particular game the agents play, namely the consensus game, but rather on the structure and dynamics that considering multiple relations (afterwards called multiplex relations) induces. In fact, we wanted to keep the game as simple as possible, so that it would be quite neutral on the phenomenon at hands. To this purpose, the main motivation is just to consider more realistic accounts of social phenomena. Our proposal goes in that direction by suggesting and defending that the concomitant belonging to several relations is closer to reality than to consider a single relation in which the various social bonds are mixed together, summarised, and to a great extent simplified, since the nature and features of relations (family, friends, colleagues, etc.) is quite different and allows diverse explorations when we are considering options in our decisions concerning or influenced by social options. Essentially, we don't pretend that our results are of direct use for extrapolation to real societies any more than Schelling would claim that his model is an accurate model of human racial segregation. Or, making the same case, we acknowledge that the Prisoner's Dilemma game, widely used as an instance of Public Goods games, and with substantial potential for depicting real situations, is not a situation in which real people find themselves frequently. What we do argue is that our use of structured social networks that exhibit scale-free properties is quite common in real world, as thoroughly reported in the literature. Our use of scale-free networks in concomitancy with other networks tries to provide even more realism and increases the applicability of our results, be those networks regular (or lattices, e.g. a common situation in classrooms, football stadiums, etc) or scale-free themselves (seeking to represent the several but not coincidental relations found in true society, both on- and off-line). Possibly, some more types of relations should have been covered, but our commitment was to take this set and make the study as exhaustive as possible.

\item[B:] To answer 2b, we have rewritten the conclusions in order to best emphasize the contributions and the significance of the results we obtained.

\item[C:] Finally, to address 3b, we performed a thorough revision and hopefully have removed all the typos and grammatical mistakes.
\end{itemize}

\end{document}

